\documentclass{article}
\usepackage{amsfonts}
\usepackage{amsmath}
\usepackage{amssymb}
\usepackage{mathrsfs}
\usepackage{fancyheadings}
\usepackage{graphicx}
\usepackage[top=1.25in, bottom=1.25in, left=1.0in, right=1.0in]{geometry}
\usepackage{hyperref}
\usepackage{tikz-cd}

\pagestyle{fancy}

\lhead{Papa Rudin Notes}

\hypersetup{colorlinks=true,linkcolor=blue,urlcolor=blue} 
\setcounter{tocdepth}{3}

\newcounter{topicnumber}[section]
\renewcommand{\thesubsubsection}{\thesection.\thetopicnumber}
\setcounter{tocdepth}{2}

\newcommand{\steptopic}[1][1]{\addtocounter{topicnumber}{#1}}
\newcommand{\material}[1]{%
\subsection*{#1}%
\addcontentsline{toc}{subsection}{#1}%
}
\newenvironment{topic}[1]{%
\steptopic%
\subsubsection{#1}%
\begin{itemize}%
}{%
\end{itemize}%
}

\newcommand{\term}[1]{{\bf #1}}
\newcommand{\remark}{\item}
\newcommand{\powerset}{\mathscr{P}}

\begin{document}
\pagenumbering{roman}
\tableofcontents
\newpage
\pagenumbering{arabic}


\section{Abstract Integration}

\material{The Concept of Measurability}

% 1.1
% 1.2
\steptopic[2]

% 1.3
\begin{topic}{Definition}
\remark $\mathfrak{M} \subseteq \powerset(X)$, containing $X$, is a $\sigma$-algebra if $\mathfrak{M}$ is closed under complementation and countable unions.
\remark $(X, \mathfrak{M})$ is a \term{measurable space}; elements of $\mathfrak{M}$ are \term{measurable sets}.
\remark $f : (X, \mathfrak{M}) \to (Y, \tau)$ is \term{measurable} if open sets have measurable preimages ($\tau$ a topology).
\end{topic}

{\em Note}: Instead of $(X, \mathfrak{M})$ we just refer to $X$ as the measurable space.

% 1.4
% 1.5
\steptopic[2]

% 1.6
\begin{topic}{Comments on Definition 1.3}
\remark $\emptyset \in \mathfrak{M}$.
\remark Finite unions are in $\mathfrak{M}$.
\remark $\mathfrak{M}$ is closed under finite and countable intersection.
\remark $\mathfrak{M}$ is closed under set subtraction.
\end{topic}

% 1.7
\begin{topic}{Composition with Continuous Functions}
\remark $X$ a measurable space, $f : X \to Y$ measurable, $g : Y \to Z$ continuous: $g \circ f$ is measurable.
\end{topic}

% 1.8
\begin{topic}{Continuous Image of Cartesian Product of Measurable Functions.}
\remark $u$, $v$ real measurable functions on $X$, $\phi$ continuous image of the plane into a topological space $Y$:  $\phi(u(x), v(x))$ is measurable.
\end{topic}

% 1.9
\begin{topic}{Creating Measurable Functions}
\remark If $u$, $v$ are real measurable then $f = u + iv$ is complex measurable.
\remark If $f = u + iv$ is complex measurable then $u$, $v$, and $|f|$ are real measurable.
\remark If $f$ and $g$ are complex measurable then so are $f + g$ and $fg$.
\remark Characteristic functions of measurable sets are measurable functions.
\remark If $f$ is complex measurable then there is a complex measurable function $\alpha$ with $|\alpha| = 1$ and $f = \alpha |f|$.
\end{topic}

% 1.10
\begin{topic}{$\sigma$-Algebra Generated by a Set}
\remark $\mathscr{F} \subseteq \powerset(X)$ is contained in some smallest $\sigma$-algebra $\mathfrak{M}^\ast$.
\end{topic}

% 1.11
\begin{topic}{Borel Sets}
\remark The \term{Borel Sets}, $\mathfrak{B}$, is the $\sigma$-algebra generated by the topology of a space.
\remark $G_\delta$ sets are countable intersections of open sets.
\remark $F_\sigma$ sets are countable unions of closed sets.
\remark Borel measurable functions are called \term{Borel mappings} or \term{Borel functions}.
\remark {\em Every} continuous function is Borel measurable.
\end{topic}

% 1.12
\begin{topic}{$\sigma$-Algebras Associated with a Function}
\remark $\mathfrak{M}$ a $\sigma$-algebra on $X$, $Y$ a topological space, $f : X \to Y$ a function.
\remark $\Omega = \{ E \subseteq Y : f^{-1}(E) \in \mathfrak{M} \}$ is a $\sigma$-algebra on $Y$.
\remark If $f$ is measurable, $E$ Borel in $Y$, then $f^{-1}(E) \in \mathfrak{M}$.
\remark If $Y = [-\infty, \infty]$ and $f^{-1}((a, \infty]) \in \mathfrak{M}$ for all $\alpha \in \mathbb{R}$ then $f$ is measurable.
\remark If $f$ is measurable, $Z$ a topological space, $g : Y \to Z$ Borel, then $g \circ f : X \to Z$ is measurable.
\end{topic}

% 1.13
\steptopic

% 1.14
\begin{topic}{Supremum and Limit Supremum of Measurable Functions}
\remark If $f_n : X \to [-\infty, \infty]$ are measurable then so are $\sup f_n$ and $\limsup f_n.$
\remark The limit of pointwise convergent sequence of complex measurable functions is measurable.
\remark $f$, $g$ measurable then so are $\max\{f, g\}$ and $\min\{f,g\}$.
\end{topic}

% 1.15
\begin{topic}{Positive and Negative parts of $f$}
\remark $f^+ = \max\{f, 0\}$ is the \term{positive part} of $f$ and $f^- = -\min\{f, 0\}$ is the \term{negative part}.
\remark $|f| = f^+ + f^-$ and $f = f^+ - f^-$.
\remark If $f = g - h$, $g \geq 0$ and $h \geq 0$ then $f^+ \leq g$ and $f^- \leq h$.
\end{topic}


\material{Simple Functions}

% 1.16
\begin{topic}{Definition}
\remark $s$, complex measurable on $X$, is \term{simple} if its range is finite. If $s(X) = \{\alpha_1, \ldots, \alpha_n\}$ then $$s = \sum_{i=1}^n \alpha_i \chi_{A_i},~~A_i = s^{-1}(\alpha_i).$$
\remark $s$ is measurable if and only if each $A_i$ is.
\end{topic}

% 1.17
\begin{topic}{Approximation by Simple Functions}
\remark If $f : X \to [0, \infty]$ is measurable then there exists measurable, simple functions $s_n$ on $X$ such that $0 \leq s_1 \leq \ldots \leq f$ and $s_n(x) \to f(x)$ as $n \to \infty$ for all $x \in X$.
\end{topic}


\material{Elementary Properties of Measures}

% 1.18
\begin{topic}{Definition}
\remark A \term{positive measure} is a function from a $\sigma$-algebra $\mathfrak{M}$ to $[0, \infty]$ which is \term{countably additive}: i.e. $$\mu\left(\bigcup_{n=1}^\infty A_i\right) = \sum_{i=1}^n \mu(A_i)$$ when $A_i$ are pairwise disjoint members of $\mathfrak{M}$.
\remark A measurable space equipped with a measure is a \term{measure space}.
\remark A \term{complex measure} is a complex-value countably additive function on a $\sigma$-algebra.
\end{topic}

% 1.19
\begin{topic}{Basic Properties of a Positive Measure $\mu$}
\remark $\mu(\emptyset) = 0$.
\remark $\mu(A_1 \cup \cdots \cup A_n) = \mu(A_1) + \cdots + \mu(A_n)$ if the $A_i$ are pairwise disjoint members of $\mathfrak{M}$.
\remark $A \subseteq B$ implies $\mu(A) \leq \mu(B)$ for $A, B \in \mathfrak{M}$.
\remark If $A_n \in \mathfrak{M}$ such that $A_1 \subseteq A_2 \subseteq A_3 \subseteq \cdots$ then $\mu(A_n) \to \mu\left(\cup_{n=1}^\infty A_n\right)$.
\remark If $A_n \in \mathfrak{M}$ such that $A_1 \supseteq A_2 \supseteq A_3 \supseteq \cdots$ and $\mu(A_1) < \infty$ then $\mu(A_n) \to \mu\left(\cap_{n=1}^\infty A_n\right)$.
\end{topic}

% 1.20
\begin{topic}{Measure Space Examples}
\remark \term{counting measure}: $\mu(E) = |E|$ if $|E| < \infty$ and $\mu(E) = \infty$ otherwise.
\remark \term{unit mass at $x_0$}: $\mu(E) = 1$ if $x_0 \in E$ and $\mu(E) = 0$ otherwise.
\end{topic}

% 1.21
\steptopic


\material{Arithmetic in $[0, \infty]$}

% 1.22
\begin{topic}{Definition}
\remark $a + \infty = \infty + a = \infty$
\remark $a \cdot \infty = \infty \cdot a = \begin{cases}
\infty&a \in (0, \infty]\\
0&a = 0
\end{cases}$
\remark With $0 \cdot \infty = 0$ we have commutativity, associativity, and distributivity.
\remark Cancellation: $a + b = a + c \implies b = c$ only if $a \neq \infty$; $ab = ac \implies b = c$ only if $a \in (0, \infty)$.
\remark $0 \leq a_1 \leq a_2 \leq \cdots$, $0 \leq b_1 \leq b_2 \leq \cdots$ with $a_n \to a$ and $b_n \to b \implies a_n b_n \to ab$.
\end{topic}


\material{Integration of Positive Functions on $(X, \mathfrak{M}, \mu)$}

% 1.23
\begin{topic}{Definition}
\remark $s : X \to [0, \infty]$ simple and measurable with $s(X) = \{\alpha_1, \ldots, \alpha_n\}$. For $E \in \mathfrak{M}$ define $$\int_E s\,d\mu = \sum_{i=1}^n \alpha_i \mu(A_i \cap E),~~A_i = s^{-1}(\alpha_i).$$
\remark If $f : X \to [0, \infty]$ is measurable then for $E \in \mathfrak{M}$ define the \term{Lebesgue Integral of $f$ over $E$} by $$\int_E f\,d\mu = \sup \int_E s\,d\mu,$$ where the supremum is taken over all nonnegative measurable simple functions dominated by $f$.
\end{topic}

% 1.24
\begin{topic}{Basic Properties of Lebesgue Integrals}
\remark $0 \leq f \leq g$ implies $\int_E f\,d\mu \leq \int_E g\,d\mu$.
\remark $A \subseteq B$ and $f \geq 0$ implies $\int_A f\,d\mu \leq \int_B f\,d\mu$.
\remark If $f \geq 0$ and $c \in [0, \infty)$ then $\int_E cf\,d\mu = c \int_E f\,d\mu$.
\remark If $f \equiv 0$ on $E$ then $\int_E f\,d\mu = 0$ even if $\mu(E) = \infty$.
\remark If $\mu(E) = 0$ then $\int_E f\,d\mu = 0$ if if $f \equiv \infty$ on $E$.
\remark If $f \geq 0$ then $\int_E f\,d\mu = \int_X \chi_E f d\mu$.
\end{topic}

% 1.25
\begin{topic}{Basic Properties of the Lebesgue Integral of Simple Functions}
\remark If $s$ is a nonnegative measurable simple function then $\varphi : \mathfrak{M} \to [0, \infty]$ sending $E$ to $\int_E s\,d\mu$ is a measure.
\remark If $s$ and $t$ are nonnegative measurable simple functions then $\int_X (s+t)\,d\mu = \int_X s\,d\mu + \int_X t\,d\mu$.
\end{topic}

% 1.26
\begin{topic}{Lebesgue's Monotone Convergence Theorem}
\remark If $f_n : X \to [0, \infty]$ are measurable functions such both $\{f_n(x)\}$ is non-decreasing $f_n(x) \to f(x)$ hold for every $x \in X$ then $f$ is measurable and $$\int_X f_n\,d\mu \to \int_X f\,d\mu.$$
\end{topic}

% 1.27
\begin{topic}{Interchange of Summation and Integration}
\remark If $f_n : X \to [0, \infty]$ are measurable and $f(x) = \sum_{n=1}^\infty f_n(x)$ then $$\int_X f\,d\mu = \sum_{n=1}^\infty f_n\,d\mu.$$
\end{topic}

% 1.28
\begin{topic}{Fatou's Lemma}
\remark If $f_n : X \to [0, \infty]$ are measurable then $$\int_X \left(\liminf_{n \to \infty} f_n\right)d\mu \leq \liminf_{n \to \infty} \int_X f\,d\mu.$$
\end{topic}

% 1.29
\begin{topic}{Change of Measure}
\remark If $f : X \to [0, \infty]$ is measurable then $\varphi : \mathfrak{M} \to [0, \infty]$ sending $E$ to $\int_E f\,d\mu$ is a measure and $$\int_X g\,d\varphi = \int_X gf\,d\mu.$$ Sometimes this is written as $d\varphi = f\,d\mu$, although no independent meaning is given to these symbols.
\end{topic}


\material{Integration of Complex Functions on $(X, \mathfrak{M}, \mu)$}


% 1.30
\begin{topic}{Definition}
\remark The \term{Lebesgue Integrable Functions} or \term{Summable Functions} with respect to $\mu$, denoted by $L^1(\mu)$ is the collection of all complex measurable functions $f$ on $X$ such that $\int_X |f|\,d\mu < \infty$.
\end{topic}

% 1.31
\begin{topic}{Definition}
\remark If $f = u + iv$ with $u$, $v$ real measurable functions and $f \in L^1(\mu)$ then for $E \in \mathfrak{M}$: $$\int_E f\,d\mu = \left(\int_E u^+\,d\mu - \int_E u^-\,d\mu\right) + i \left(\int_E v^+\,d\mu - \int_E v^-\,d\mu\right).$$
\remark It is useful define the integral of a function $f : X \to [-\infty, \infty]$ to be $$\int_E f\,d\mu = \int_E f^+\,d\mu - \int_E f^-\,d\mu$$ for $E \in \mathfrak{M}$ and provided only one term on the right is infinite.
\end{topic}

% 1.32
\begin{topic}{Linearity of $L^1(\mu)$}
\remark For $f, g \in L^1(\mu)$ and $\alpha, \beta \in \mathbb{C}$ we have $\alpha f + \beta g \in \mathbb{L^1(\mu)}$ and $$\int_X (\alpha f + \beta g)\,d\mu = \alpha \int_X f\,d\mu + \beta \int_X g\,d\mu.$$
\end{topic}

% 1.33
\begin{topic}{Interchange of Modulus and Integration}
\remark $\left|\int_X f\,d\mu\right| \leq \int_X |f|\,d\mu$ for $f \in L^1(\mu)$.
\end{topic}

% 1.34
\begin{topic}{Lebesgue's Dominated Convergence Theorem}
\remark $f_n$ are complex measurable functions such that $f(x) = \lim_{n \to \infty} f_n(x)$ exists for all $x \in X$. If $$|f_n(x)| \leq g(x),~~\text{for all}~n \in \mathbb{N}$$ for some $g \in L^1(\mu)$ then $f \in L^1(\mu)$, $$\lim_{n \to \infty} \int_X |f_n - f|\,d\mu = 0,$$ and $$\lim_{n \to \infty} \int_X f_n\,d\mu = \int_X f\,d\mu.$$
\end{topic}


\material{The Role Played by Sets of Measure Zero}

% 1.35
\begin{topic}{Definition}
\remark If $\mu$ is a measure on a $\sigma$-algebra $\mathfrak{M}$, $E \in \mathfrak{M}$, then a statement $P$ holds \term{almost everywhere} (a.e.) on E if there exists $N \subseteq E$ with $\mu(N) = 0$ such that $P$ is true on $E \setminus N$.
\remark Example: for $f, g$ measurable if $\mu(\{ x : f(x) \neq g(x) \}) = 0$ then $f = g$ a.e. and we write $f \sim g$. $\sim$ is an equivalence relation and if $f \sim g$ then for $E \in \mathfrak{M}$ we have $\int_E f\,d\mu = \int_E g\,d\mu$. Thus sets of measure zero are negligible with respect to integration.
\remark {\em Note:} It is {\em not} the case that a subset of a negligible set is negligible as it may not even be measurable!
\end{topic}

% 1.36
\begin{topic}{Existence of Completions}
\remark $(X, \mathfrak{M}, \mu)$ is a measure space. Define $\mathfrak{M}^\ast$ to be all $E \subseteq X$ such that $A \subseteq E \subseteq B$ for $A, B \in \mathfrak{M}$ such that $\mu(B \setminus A) = 0$. Defining $\mu(E) = \mu(A)$, $(X, \mathfrak(M)^\ast, \mu)$ is a measure space,
\remark The extended $\mu$ is \term{complete} as all subsets of negligible sets are measurable.
\remark $\mathfrak{M}^\ast$ is the \term{$\mu$-completion} of $\mathfrak{M}$.
\end{topic}

% 1.37
\begin{topic}{Expanding the Definition of What is a Measurable Function}
\remark Since integration is agnostic to functions equal a.e., we now call $f$ defined on $E \in \mathfrak{M}$ \term{measurable on $X$} if $\mu(E^c) = 0$ and $f^{-1}(V) \cap E$ is measurable for every open set $V$.
\remark In the above, we can define $f \equiv 0$ on $E^c$ to get a measurable function on $X$.
\end{topic}

% 1.38
\begin{topic}{Lebesgue's Dominated Convergence Theorem with Negligible Sets}
\remark $f_n$ complex measurable functions defined a.e.\ on $X$ such that $$\sum_{n=1}^\infty \int_X |f_n|\,d\mu < \infty.$$ Then $f(x) = \sum_{n=1}^\infty f_n(x)$ converges for almost all $x$ and $f \in L^1(\mu)$ with $$\int_X f\,d\mu = \sum_{n=1}^\infty \int_X f_n\,d\mu.$$
\end{topic}

% 1.39
\begin{topic}{Integration and Properties That Hold Almost Everywhere}
\remark If $f : X \to [0, \infty]$ measurable, $E \in \mathfrak{M}$ with $\int_E f\,d\mu = 0$ then $f = 0$ a.e.\ on $E$.
\remark If $f \in L^1(\mu)$ with $\int_E f\,d\mu = 0$ for every $E \in \mathfrak{M}$ then $f = 0$ a.e.\ on $X$.
\remark If $f \in L^1(\mu)$ and $$\left|\int_X f\,d\mu\right| = \int_X |f|\,d\mu$$ then there exists $\alpha \in \mathbb{C}$ such that $\alpha f = |f|$ a.e.\ on $X$.
\end{topic}

% 1.40
\begin{topic}{Averages Lying in a Closed Set}
\remark If $\mu(X) < \infty$, $f \in L^1(\mu)$, $S \subseteq \mathbb{C}$ is closed, and the averages $$A_E(f) = \dfrac{1}{\mu(E)} \int_E f\,d\mu$$ lie in $S$ for every $E \in \mathfrak{M}$ with positive measure then $f(x) \in S$ for almost all $x \in X$.
\end{topic}

% 1.41
\begin{topic}{Finite Set Membership}
\remark If $E_k \subseteq X$ are measurable with $\sum_{k=1}^\infty \mu(E_k) < \infty$ then almost all $x \in X$ lie in finitely many $E_k$.
\end{topic}


\newpage
\section{Positive Borel Measures}

\end{document}
