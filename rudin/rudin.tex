\documentclass{article}
\usepackage{amsfonts}
\usepackage{amsmath}
\usepackage{amssymb}
\usepackage{mathrsfs}
\usepackage{fancyheadings}
\usepackage[top=1.25in, bottom=1.25in, left=1.0in, right=1.0in]{geometry}
\usepackage{hyperref}

\pagestyle{fancy}
\lhead{Papa Rudin Notes}
\hypersetup{colorlinks=true,linkcolor=blue,urlcolor=blue} 
\setcounter{tocdepth}{3}

\newcounter{topicnumber}[section]
\renewcommand{\thesubsubsection}{\thesection.\thetopicnumber}
\setcounter{tocdepth}{2}
\newcounter{examplenumber}[topicnumber]

\newcommand{\steptopic}[1][1]{\addtocounter{topicnumber}{#1}}
\newcommand{\material}[1]{%
\subsection*{#1}%
\addcontentsline{toc}{subsection}{#1}%
}
\newenvironment{topic}[1]{%
\steptopic%
\subsubsection{#1}%
\begin{itemize}%
}{%
\end{itemize}%
}

\newcommand{\term}[1]{{\bf #1}}
\newcommand{\setup}{\item[\null]}
\newcommand{\remark}{\item}
\newcommand{\note}{\item[{\em Note:}]}
\newcommand{\example}{%
\addtocounter{examplenumber}{1}%
\item[{\em Example \theexamplenumber}]%
}
\newcommand{\powerset}{\mathscr{P}}

% https://tex.stackexchange.com/questions/88281/how-to-change-font-for-the-integral-symbol
\DeclareSymbolFont{rudin-largesymbols}{OMX}{mdbch}{m}{n}
\DeclareMathSymbol{\intop}{\mathop}{rudin-largesymbols}{82}
\DeclareMathOperator{\vol}{vol\,}

\begin{document}
\pagenumbering{roman}
\tableofcontents
\newpage
\pagenumbering{arabic}


\section{Abstract Integration}

% 1.1
\steptopic

\material{The Concept of Measurability}

% 1.2
\begin{topic}{Definition}
\remark $\tau \subseteq \powerset(X)$, containing both $\emptyset$ and $X$, is a \term{topology} if it is closed under finite intersections and arbitrary unions.
\remark $(X, \tau)$ is a \term{topological space} and the members of $\tau$ are \term{open sets}.
\remark $f : (X, \tau_X) \to (Y, \tau_Y)$ is \term{continuous} if open sets have open preimages.
\end{topic}

% 1.3
\begin{topic}{Definition}
\remark $\mathfrak{M} \subseteq \powerset(X)$, containing $X$, is a \term{$\sigma$-algebra} if it is closed under complementation and countable unions.
\remark $(X, \mathfrak{M})$ is a \term{measurable space}; elements of $\mathfrak{M}$ are \term{measurable sets}.
\remark $f : (X, \mathfrak{M}) \to (Y, \tau)$ is \term{measurable} if open sets have measurable preimages ($\tau$ a topology).
\note Instead of $(X, \mathfrak{M})$ we just refer to $X$ as the measurable space.
\end{topic}

% 1.4
% 1.5
\steptopic[2]

% 1.6
\begin{topic}{Comments on Definition 1.3}
\remark $\emptyset \in \mathfrak{M}$.
\remark Finite unions are in $\mathfrak{M}$.
\remark $\mathfrak{M}$ is closed under finite and countable intersection.
\remark $\mathfrak{M}$ is closed under set subtraction.
\end{topic}

% 1.7
\begin{topic}{Composition with Continuous Functions}
\remark $f$ measurable, $g$ continuous: $g \circ f$ is measurable.
\end{topic}

% 1.8
\begin{topic}{Continuous Image of Cartesian Product of Measurable Functions.}
\remark $u$, $v$ real, measurable functions; $\phi$ continuous on the plane: $\phi(u(x), v(x))$ is measurable.
\end{topic}

% 1.9
\begin{topic}{Creating Measurable Functions}
\remark If $u$, $v$ are real measurable then $f = u + iv$ is complex measurable.
\remark If $f = u + iv$ is complex measurable then $u$, $v$, and $|f|$ are real measurable.
\remark If $f$ and $g$ are complex measurable then so are $f + g$ and $fg$.
\remark Characteristic functions of measurable sets are measurable functions.
\remark If $f$ is complex measurable then there is a complex measurable function $\alpha$ with $|\alpha| = 1$ and $f = \alpha |f|$.
\end{topic}

% 1.10
\begin{topic}{$\sigma$-Algebra Generated by a Set}
\remark $\mathscr{F} \subseteq \powerset(X)$ is contained in some smallest $\sigma$-algebra $\mathfrak{M}^\ast$.
\end{topic}

% 1.11
\begin{topic}{Borel Sets}
\remark The \term{Borel Sets}, $\mathfrak{B}$, is the $\sigma$-algebra generated by the topology of a space.
\remark $G_\delta$ sets are countable intersections of open sets.
\remark $F_\sigma$ sets are countable unions of closed sets.
\remark Borel measurable functions are called \term{Borel mappings} or \term{Borel functions}.
\remark {\em Every} continuous function is Borel measurable.
\end{topic}

% 1.12
\begin{topic}{$\sigma$-Algebras Associated with a Function}
\setup $\mathfrak{M}$ a $\sigma$-algebra on $X$, $Y$ a topological space, $f : X \to Y$ a function:
\remark $\Omega = \{ E \subseteq Y : f^{-1}(E) \in \mathfrak{M} \}$ is a $\sigma$-algebra on $Y$.
\remark If $f$ is measurable, $E$ Borel in $Y$, then $f^{-1}(E) \in \mathfrak{M}$.
\remark If $Y = [-\infty, \infty]$ and $f^{-1}((a, \infty]) \in \mathfrak{M}$ for all $\alpha \in \mathbb{R}$ then $f$ is measurable.
\remark If $f$ is measurable, $Z$ a topological space, $g : Y \to Z$ Borel, then $g \circ f : X \to Z$ is measurable.
\end{topic}

% 1.13
\steptopic

% 1.14
\begin{topic}{Supremum and Limit Supremum of Measurable Functions}
\remark If $f_n : X \to [-\infty, \infty]$ are measurable then so are $\sup f_n$ and $\limsup f_n.$
\remark The limit of pointwise convergent sequence of complex measurable functions is measurable.
\remark $f$, $g$ measurable then so are $\max\{f, g\}$ and $\min\{f,g\}$.
\end{topic}

% 1.15
\begin{topic}{Positive and Negative parts of $f$}
\remark $f^+ = \max\{f, 0\}$ is the \term{positive part} of $f$ and $f^- = -\min\{f, 0\}$ is the \term{negative part}.
\remark $|f| = f^+ + f^-$ and $f = f^+ - f^-$.
\remark If $f = g - h$, $g \geq 0$ and $h \geq 0$ then $f^+ \leq g$ and $f^- \leq h$.
\end{topic}


\material{Simple Functions}

% 1.16
\begin{topic}{Definition}
\remark $s$, complex measurable on $X$, is \term{simple} if its range is finite. If $s(X) = \{\alpha_1, \ldots, \alpha_n\}$ then $$s = \sum_{i=1}^n \alpha_i \chi_{A_i},~~A_i = s^{-1}(\alpha_i).$$
\remark $s$ is measurable if and only if each $A_i$ is.
\end{topic}

% 1.17
\begin{topic}{Approximation by Simple Functions}
\remark If $f : X \to [0, \infty]$ is measurable then there exists measurable, simple functions $s_n$ on $X$ such that $0 \leq s_1 \leq \ldots \leq f$ and $s_n(x) \to f(x)$ as $n \to \infty$ for all $x \in X$.
\end{topic}


\material{Elementary Properties of Measures}

% 1.18
\begin{topic}{Definition}
\remark A \term{positive measure} is a function from a $\sigma$-algebra $\mathfrak{M}$ to $[0, \infty]$ which is \term{countably additive}: i.e. $$\mu\left(\bigcup_{n=1}^\infty A_i\right) = \sum_{i=1}^n \mu(A_i)$$ when $A_i$ are pairwise disjoint members of $\mathfrak{M}$.
\remark A measurable space equipped with a measure is a \term{measure space}.
\remark A \term{complex measure} is a complex-valued countably additive function on a $\sigma$-algebra.
\end{topic}

% 1.19
\begin{topic}{Basic Properties of a Positive Measure $\mu$}
\remark $\mu(\emptyset) = 0$.
\remark $\mu(A_1 \cup \cdots \cup A_n) = \mu(A_1) + \cdots + \mu(A_n)$ if the $A_i$ are pairwise disjoint members of $\mathfrak{M}$.
\remark $A \subseteq B$ implies $\mu(A) \leq \mu(B)$ for $A, B \in \mathfrak{M}$.
\remark If $A_n \in \mathfrak{M}$ such that $A_1 \subseteq A_2 \subseteq A_3 \subseteq \cdots$ then $\mu(A_n) \to \mu\left(\cup_{n=1}^\infty A_n\right)$.
\remark If $A_n \in \mathfrak{M}$ such that $A_1 \supseteq A_2 \supseteq A_3 \supseteq \cdots$ and $\mu(A_1) < \infty$ then $\mu(A_n) \to \mu\left(\cap_{n=1}^\infty A_n\right)$.
\end{topic}

% 1.20
\begin{topic}{Measure Space Examples}
\remark \term{counting measure}: $\mu(E) = |E|$ if $|E| < \infty$ and $\mu(E) = \infty$ otherwise.
\remark \term{unit mass at $x_0$}: $\mu(E) = 1$ if $x_0 \in E$ and $\mu(E) = 0$ otherwise.
\end{topic}

% 1.21
\steptopic


\material{Arithmetic in $[0, \infty]$}

% 1.22
\begin{topic}{Definition}
\remark $a + \infty = \infty + a = \infty$
\remark $a \cdot \infty = \infty \cdot a = \begin{cases}
\infty&a \in (0, \infty]\\
0&a = 0
\end{cases}$
\remark With $0 \cdot \infty = 0$ we have commutativity, associativity, and distributivity.
\remark Cancellation: $a + b = a + c \implies b = c$ only if $a \neq \infty$; $ab = ac \implies b = c$ only if $a \in (0, \infty)$.
\remark $0 \leq a_1 \leq a_2 \leq \cdots$, $0 \leq b_1 \leq b_2 \leq \cdots$ with $a_n \to a$ and $b_n \to b \implies a_n b_n \to ab$.
\end{topic}


\material{Integration of Positive Functions on $(X, \mathfrak{M}, \mu)$}

% 1.23
\begin{topic}{Definition}
\remark $s : X \to [0, \infty]$ simple and measurable with $s(X) = \{\alpha_1, \ldots, \alpha_n\}$. For $E \in \mathfrak{M}$ define $$\int_E s\,d\mu = \sum_{i=1}^n \alpha_i \mu(A_i \cap E),~~A_i = s^{-1}(\alpha_i).$$
\remark If $f : X \to [0, \infty]$ is measurable then for $E \in \mathfrak{M}$ define the \term{Lebesgue Integral of $f$ over $E$} by $$\int_E f\,d\mu = \sup \int_E s\,d\mu,$$ where the supremum is taken over all nonnegative measurable simple functions dominated by $f$.
\end{topic}

% 1.24
\begin{topic}{Basic Properties of Lebesgue Integrals}
\remark $0 \leq f \leq g$ implies $\int_E f\,d\mu \leq \int_E g\,d\mu$.
\remark $A \subseteq B$ and $f \geq 0$ implies $\int_A f\,d\mu \leq \int_B f\,d\mu$.
\remark If $f \geq 0$ and $c \in [0, \infty)$ then $\int_E cf\,d\mu = c \int_E f\,d\mu$.
\remark If $f \equiv 0$ on $E$ then $\int_E f\,d\mu = 0$ even if $\mu(E) = \infty$.
\remark If $\mu(E) = 0$ then $\int_E f\,d\mu = 0$ if if $f \equiv \infty$ on $E$.
\remark If $f \geq 0$ then $\int_E f\,d\mu = \int_X \chi_E f d\mu$.
\end{topic}

% 1.25
\begin{topic}{Basic Properties of the Lebesgue Integral of Simple Functions}
\remark If $s$ is a nonnegative measurable simple function then $\varphi : \mathfrak{M} \to [0, \infty]$ sending $E$ to $\int_E s\,d\mu$ is a measure.
\remark If $s$ and $t$ are nonnegative measurable simple functions then $\int_X (s+t)\,d\mu = \int_X s\,d\mu + \int_X t\,d\mu$.
\end{topic}

% 1.26
\begin{topic}{Lebesgue's Monotone Convergence Theorem}
\remark If $f_n : X \to [0, \infty]$ is a (point-wise) non-decreasing sequence of measurable functions for which $f_n(x) \to f(x)$ for every $x \in X$ then $f$ is measurable and $$\int_X f_n\,d\mu \to \int_X f\,d\mu.$$
\end{topic}

% 1.27
\begin{topic}{Interchange of Summation and Integration}
\remark If $f_n : X \to [0, \infty]$ are measurable and $f(x) = \sum_{n=1}^\infty f_n(x)$ then $$\int_X f\,d\mu = \sum_{n=1}^\infty f_n\,d\mu.$$
\end{topic}

% 1.28
\begin{topic}{Fatou's Lemma}
\remark If $f_n : X \to [0, \infty]$ are measurable then $$\int_X \left(\liminf_{n \to \infty} f_n\right)d\mu \leq \liminf_{n \to \infty} \int_X f\,d\mu.$$
\end{topic}

% 1.29
\begin{topic}{Change of Measure}
\remark If $f : X \to [0, \infty]$ is measurable then $\varphi : \mathfrak{M} \to [0, \infty]$ sending $E$ to $\int_E f\,d\mu$ is a measure and $$\int_X g\,d\varphi = \int_X gf\,d\mu.$$ Sometimes this is written as $d\varphi = f\,d\mu$, although no independent meaning is given to these symbols.
\end{topic}


\material{Integration of Complex Functions on $(X, \mathfrak{M}, \mu)$}

% 1.30
\begin{topic}{Definition}
\remark The \term{Lebesgue Integrable Functions} or \term{Summable Functions} with respect to $\mu$, denoted by $L^1(\mu)$ is the collection of all complex measurable functions $f$ on $X$ such that $\int_X |f|\,d\mu < \infty$.
\end{topic}

% 1.31
\begin{topic}{Definition}
\remark If $f = u + iv$ with $u$, $v$ real measurable functions and $f \in L^1(\mu)$ then for $E \in \mathfrak{M}$: $$\int_E f\,d\mu = \left(\int_E u^+\,d\mu - \int_E u^-\,d\mu\right) + i \left(\int_E v^+\,d\mu - \int_E v^-\,d\mu\right).$$
\remark It is useful define the integral of a function $f : X \to [-\infty, \infty]$ to be $$\int_E f\,d\mu = \int_E f^+\,d\mu - \int_E f^-\,d\mu$$ for $E \in \mathfrak{M}$ and provided only one term on the right is infinite.
\end{topic}

% 1.32
\begin{topic}{Linearity of $L^1(\mu)$}
\remark For $f, g \in L^1(\mu)$ and $\alpha, \beta \in \mathbb{C}$ we have $\alpha f + \beta g \in L^1(\mu)$ and $$\int_X (\alpha f + \beta g)\,d\mu = \alpha \int_X f\,d\mu + \beta \int_X g\,d\mu.$$
\end{topic}

% 1.33
\begin{topic}{Interchange of Modulus and Integration}
\remark $\left|\int_X f\,d\mu\right| \leq \int_X |f|\,d\mu$ for $f \in L^1(\mu)$.
\end{topic}

% 1.34
\begin{topic}{Lebesgue's Dominated Convergence Theorem}
\remark $f_n$ are complex measurable functions such that $f(x) = \lim_{n \to \infty} f_n(x)$ exists for all $x \in X$. If $$|f_n(x)| \leq g(x),~~\text{for all}~n \in \mathbb{N}$$ for some $g \in L^1(\mu)$ then $f \in L^1(\mu)$, $$\lim_{n \to \infty} \int_X |f_n - f|\,d\mu = 0,$$ and $$\lim_{n \to \infty} \int_X f_n\,d\mu = \int_X f\,d\mu.$$
\end{topic}


\material{The Role Played by Sets of Measure Zero}

% 1.35
\begin{topic}{Definition}
\remark If $\mu$ is a measure on a $\sigma$-algebra $\mathfrak{M}$, $E \in \mathfrak{M}$, then a statement $P$ holds \term{almost everywhere} (a.e.) on E if there exists $N \subseteq E$ with $\mu(N) = 0$ such that $P$ is true on $E \setminus N$.
\example If $f$ and $g$ are measurable and $\mu(\{ x : f(x) \neq g(x) \}) = 0$ then $f = g$ a.e., written as $f \sim g$. $\sim$ is an equivalence relation and if $f \sim g$ then for $E \in \mathfrak{M}$ we have $\int_E f\,d\mu = \int_E g\,d\mu$. Thus sets of measure zero are negligible with respect to integration.
\note It is {\em not} the case that a subset of a negligible set is negligible as it may not even be measurable!
\end{topic}

% 1.36
\begin{topic}{Existence of Completions}
\remark If $(X, \mathfrak{M}, \mu)$ is a measure space, then define $\mathfrak{M}^\ast$ to be all $E \subseteq X$ such that $A \subseteq E \subseteq B$ for $A, B \in \mathfrak{M}$ such that $\mu(B \setminus A) = 0$. Defining $\mu(E) = \mu(A)$ makes $(X, \mathfrak{M}^\ast, \mu)$ a measure space,
\remark The extended $\mu$ is \term{complete} as all subsets of negligible sets are measurable.
\remark $\mathfrak{M}^\ast$ is the \term{$\mu$-completion} of $\mathfrak{M}$.
\end{topic}

% 1.37
\begin{topic}{Expanding the Definition of What is a Measurable Function}
\remark Since integration is agnostic to functions equal a.e., we now call $f$ defined on $E \in \mathfrak{M}$ \term{measurable on $X$} if $\mu(E^c) = 0$ and $f^{-1}(V) \cap E$ is measurable for every open set $V$.
\remark In the above, we can define $f \equiv 0$ on $E^c$ to get a measurable function on $X$.
\end{topic}

% 1.38
\begin{topic}{Lebesgue's Dominated Convergence Theorem with Negligible Sets}
\remark $f_n$ complex measurable functions defined a.e.\ on $X$ such that $$\sum_{n=1}^\infty \int_X |f_n|\,d\mu < \infty.$$ Then $f(x) = \sum_{n=1}^\infty f_n(x)$ converges for almost all $x$ and $f \in L^1(\mu)$ with $$\int_X f\,d\mu = \sum_{n=1}^\infty \int_X f_n\,d\mu.$$
\end{topic}

% 1.39
\begin{topic}{Integration and Properties That Hold Almost Everywhere}
\remark If $f : X \to [0, \infty]$ measurable and $E \in \mathfrak{M}$ with $\int_E f\,d\mu = 0$ then $f = 0$ a.e.\ on $E$.
\remark If $f \in L^1(\mu)$ with $\int_E f\,d\mu = 0$ for every $E \in \mathfrak{M}$ then $f = 0$ a.e.\ on $X$.
\remark If $f \in L^1(\mu)$ and $$\left|\int_X f\,d\mu\right| = \int_X |f|\,d\mu$$ then there exists $\alpha \in \mathbb{C}$ such that $\alpha f = |f|$ a.e.\ on $X$.
\end{topic}

% 1.40
\begin{topic}{Averages Lying in a Closed Set}
\remark If $\mu(X) < \infty$, $f \in L^1(\mu)$, $S \subseteq \mathbb{C}$ is closed, and the averages $$A_E(f) = \dfrac{1}{\mu(E)} \int_E f\,d\mu$$ lie in $S$ for every $E \in \mathfrak{M}$ with positive measure then $f(x) \in S$ for almost all $x \in X$.
\end{topic}

% 1.41
\begin{topic}{Finite Set Membership}
\remark If $E_k \subseteq X$ are measurable with $\sum_{k=1}^\infty \mu(E_k) < \infty$ then almost all $x \in X$ lie in finitely many $E_k$.
\end{topic}


\newpage
\section{Positive Borel Measures}

\material{Vector Spaces}

% 2.1
\begin{topic}{Definition}
\remark A \term{complex vector space} is one with complex scalars.
\remark A function $\Lambda$ between vector spaces is a \term{linear transformation} if $\Lambda(\alpha x + \beta y) = \alpha \Lambda x + \beta \Lambda y$.
\remark A \term{linear functional} is a linear transformation where the codomain is the field of scalars of the domain.
\end{topic}

% 2.2
\begin{topic}{Integration as a Linear Functional}
\remark For any positive measure $\mu$, $f \mapsto \int_X f\,d\mu$ is a linear functional on $L^1(\mu)$.
\remark If $g$ is a bounded measurable function then $f \mapsto \int_X fg\,d\mu$ is a linear functional on $L^1(\mu)$.
\remark A \term{positive linear functional} is a linear functional $\Lambda$ such that $\Lambda f \geq 0$ whenever $f \geq 0$.
\remark If $C$ is the vector space of continuous complex functions on $[0, 1]$ then $$\Lambda f = \int_0^1 f(x)\,dx$$ is a positive linear functional on $C$ (with the integral being the Riemann integral).
\end{topic}


\material{Topological Preliminaries}

% 2.3
\begin{topic}{Definitions}
\remark $E$ is \term{closed} if its complement is open.
\remark The \term{closure} of $E$, denoted $\overline{E}$, is the smallest closed set containing $E$.
\remark $K \subseteq X$ is \term{compact} if every open cover of $K$ contains a finite subcover.
\remark A \term{neighborhood} of $p \in X$ is any open set containing $p$.
\remark $X$ is \term{Hausdorff} if any two $p \neq q$ can be separated by open sets.
\remark $X$ is \term{locally compact} if every points has a neighborhood with compact closure.
\remark Recall Heine-Borel: Subsets of Euclidean space are compact exactly when they are closed and bounded. Thus $\mathbb{R}^n$ is locally compact.
\remark Recall: Every metric space is Hausdorff.
\end{topic}

% 2.4
\begin{topic}{Closed Subsets of Compact Sets}
\remark $F \subseteq K$, $F$ closed, $K$ compact. Then $F$ is compact.
\remark If $A \subseteq B$ and $B$ has compact closure then so does $A$.
\end{topic}

% 2.5
\begin{topic}{Separating a Compact Set  from a Point}
\remark If $X$ is Hausdorff with $K$ compact in $X$ then any $p \not\in K$ can be separated from $K$ by open sets.
\end{topic}

% 2.6
\begin{topic}{Intersections of Compact Sets}
\remark If $K_\alpha \subseteq X$ are compact, $X$ Hausdorff, and $\cap_\alpha K_\alpha = \emptyset$ then some finite subset has empty intersection.
\end{topic}

% 2.7
\begin{topic}{Sandwiching Sets}
\remark If $U$ is open in $X$, Hausdorff, and $K \subseteq U$ is compact then there exists $V$, open, with $K \subseteq V \subseteq \overline{V} \subseteq U$.
\end{topic}

% 2.8
\begin{topic}{Definition}
\remark $f$ is \term{lower semicontinuous} if $f^{-1}((\alpha, \infty])$ is open.
\remark $f$ is \term{upper semicontinuous} if $f^{-1}([\infty, \alpha))$ is open.
\remark $\chi_U$ is lower semicontinuous if $U$ is open.
\remark $\chi_F$ is upper semicontinuous if $F$ is closed.
\remark The supremum of any collection of lower semicontinuous functions is again lower semicontinuous.
\remark The infimum of any collection of upper semicontinuous functions is again upper semicontinuous.
\end{topic}

% 2.9
\begin{topic}{Definition}
\remark The \term{support} of $f : X \to \mathbb{C}$ is the closure of $f^{-1}(\mathbb{C} \setminus \{0\})$.
\remark $C_c(X)$ is the vector space of functions with compact support.
\end{topic}

% 2.10
\begin{topic}{Image of a Compact Set}
\remark The continuous image of a compact set is compact.
\remark The range of $f \in C_c(X)$ is compact subset of $\mathbb{C}$.
\end{topic}

% 2.11
\begin{topic}{Notation}
\remark $K \prec f$ means $K$ is compact, $0 \leq f(x) \leq 1$ and $f = 1$ on $K$.
\remark $f \prec V$ means $V$ is open and $f$'s support lies in $V$.
\remark $K \prec f \prec V$ combines the above.
\end{topic}

% 2.12
\begin{topic}{Urysohn's Lemma}
\remark For $K \subseteq V \subseteq X$ with $K$ compact, $V$ open, and both $X$ locally compact and Hausdorff, there exists $f \in C_c(X)$ with $K \prec f \prec V$.
\remark In terms of characteristic functions this means there is a continuous $f$ with $\chi_K \leq f \leq \chi_V$.
\end{topic}

% 2.13
\begin{topic}{Partition of Unity}
\remark For $X$ locally compact and Hausdorff and $V_1, \ldots, V_n$ open in $X$ and $K \subseteq V_1 \cup \cdots \cup V_n$ is compact, there exists functions $h_i \prec V_i$ with $h_1 + \cdots + h_n = 1$ on $K$.
\remark This is called a \term{partition of unity on $K$} subordinate to the cover $\{V_1, \ldots, V_n\}$.
\end{topic}


\material{The Riesz Representation Theorem}

% 2.14
\begin{topic}{The Riesz Representation Theorem}
\remark For $X$ locally compact and Hausdorff with $\Lambda$ a positive linear functional on $C_c(X)$ there exists:
\begin{enumerate}
\item a $\sigma$-algebra $\mathfrak{M}$ containing all the Borel sets of $X$;
\item a {\em unique} positive measure $\mu$ on $\mathfrak{M}$ representing $\Lambda$:
\begin{itemize}
\item $\Lambda f = \int_X f\,d\mu$ for all $f \in C_c(X)$
\item $\mu(K) < \infty$ for any compact $K$
\item $\mu(E) = \inf\,\{ \mu(V) : E \subseteq V,\;V\;\text{open}\}$ for all $E \in \mathfrak{M}$
\item $\mu(E) = \sup\,\{ \mu(K) : K \subseteq E,\;K\;\text{compact}\}$ for every $E \in \mathfrak{M}$ open or $\mu(E) < \infty$
\item $E \in \mathfrak{M}$, $\mu(E) = 0$ implies every subset of $E$ is in $\mathfrak{M}$.
\end{itemize}
\end{enumerate}
\end{topic}


\material{Regularity Properties of Borel Measures}

% 2.15
\begin{topic}{Definition}
\remark $\mu$ is a \term{Borel measure} if it is defined on the Borel sets.

Let $\mu$ be a positive Borel measure and $E \subseteq X$ be Borel. Then:
\remark $E$ is \term{outer regular} if the infimum property in 2.14 holds.
\remark $E$ is \term{inner regular} if the supremum property in 2.14 holds.
\remark $E$ is \term{regular} if both hold.
\end{topic}

% 2.16
\begin{topic}{Definition}
\remark $E$ is \term{$\sigma$-compact} if it is the countable union of compact sets.
\remark $E$ is \term{$\sigma$-finite} if it is the countable union of sets with finite measure.
\end{topic}

% 2.17
\begin{topic}{Regularity of $\sigma$-Compact Spaces}
\remark $X$ a locally compact, $\sigma$-compact Hausdorff space and $\mathfrak{M}, \mu$ are as in 2.14. Then:
\begin{itemize}
\item For $E \in \mathfrak{M}$, $\epsilon > 0$, there is $F \subseteq E \subseteq V$ with $F$ closed, $V$ open, and $\mu(V \setminus F) < \epsilon$.
\item $\mu$ is a regular Borel measure on $X$.
\item There exists an $F_\sigma A$ and a $G_\delta B$ with $A \subseteq E \subseteq B$ and $\mu(B \setminus A) = 0$.

Thus, every $E \in \mathfrak{M}$ is the union of an $F_\sigma$ and a negligible set.
\end{itemize}
\end{topic}

% 2.18
\begin{topic}{Regularity in the Presence of $\sigma$-Compact Open Sets}
\remark $X$ a locally compact Hausdorff space in which every open set is $\sigma$-compact. Then any positive Borel measure that is finite on compact sets is regular.
\end{topic}


\material{Lebesgue Measure}

% 2.19
\begin{topic}{Euclidean Spaces}
\remark $\mathbb{R}^k$ is the \term{$k$-dimension Euclidean space} with all the familiar operations.
\remark If $E \subseteq \mathbb{R}^k$ and $x \in \mathbb{R}^k$ then $E + x = \{ y + x : y \in E \}$ is a \term{translate} of $E$.
\remark A \term{$k$-cell} is a set of the form $\{ (\xi_1, \ldots, \xi_k) \in \mathbb{R}^k : \alpha_i < \xi_i < \beta_i, 1 \leq i \leq k \}$. Either inequality may be replaced with $\leq$. The \term{volume} of a $k$-cell is $\vol(W) = \prod_{i=1}^k (\beta_i - \alpha_i)$.
\remark If $a \in \mathbb{R}^k$ and $\delta > 0$ then a \term{$\delta$-box with corner at $a$} is $$Q(a, \delta) = \{ (\xi_1, \ldots, \xi_k) \in \mathbb{R}^k : \alpha_i \leq \xi_i < \alpha_i + \delta, 1 \leq i \leq k \}.$$
\remark If $P_n$ are points whose coordinates are multiples of $2^{-n}$ and $\Omega_n$ are the $2^{-n}$ boxes with corners at the elements of $P_n$ then we use the following properties:
\begin{itemize}
\item $\Omega_n$ covers $\mathbb{R}^k$ disjointly.
\item If $r < n$ and $Q^\prime \in \Omega_n$, $Q^{\prime\prime} \in \Omega_r$ then either $Q^\prime \subseteq Q^{\prime\prime}$ or $Q^\prime \cap Q^{\prime\prime} = \emptyset$.
\item $\vol{Q} = 2^{-rk}$ for $Q \in \Omega_r$ and if $n > r$ then $|P_n \cap Q| = 2^{(n-r)k}$.
\item Any non-empty open set is the countable disjoint union of elements of $\cup_{n=1}^\infty Q_n$.
\end{itemize}
\end{topic}

% 2.20
\begin{topic}{Existence of the Lebesgue Measure}
\remark There exists $(\mathbb{R}^k, \mathfrak{M}, m)$ such that
\begin{itemize}
\item $m(W) = \vol(W)$ for every $k$-cell $W$.
\item $\mathfrak{M}$ contains the Borel sets of $\mathbb{R}^k$
\item $E \in \mathfrak{M}$ iff $A \subseteq E \subseteq B$ with $A$ is $F_\sigma$, $B$ is $G_\delta$, and $m(B \setminus A) = 0$.
\item $m$ is regular.
\item $m(x + E) = m(E)$ for all $E \in \mathfrak{M}$ and $x \in \mathbb{R}^k$.
\item If $\mu$ is any positive translation-invariant Borel measure on $\mathbb{R}^k$ which is finite on compact sets then $\mu(E) = cm(E)$ for some $c \in \mathbb{R}$ and all Borel sets $E$.
\item $m(T(E)) = \Delta(T)m(E)$, $\Delta(T) \in \mathbb{R}$, for every linear transformation $T : \mathbb{R}^k \to \mathbb{R}^k$ and $E \in \mathfrak{M}$. More specifically, $\Delta(T) = 1$ if $T$ is a rotation.
\end{itemize}
\remark Elements of $\mathfrak{M}$ are \term{Lebesgue measurable} sets and $m$ is the \term{Lebesgue measure} on $\mathbb{R}^k$.
\end{topic}

% 2.21
\begin{topic}{Remarks}
\remark If $m$ is the Lebesgue measure on $\mathbb{R}^k$ we write $L^1(\mathbb{R}^k)$ instead of $L^1(m)$.
\remark Instead of $f \in L^1$ on $E$ we write $f \in L^1(E)$ (in the measure space with $m$ restricted to subsets of $E$).
\remark If $I$ is an interval in $\mathbb{R}$ and $f \in L^1(I)$ we write $\int_a^b f(x)\,dx$ instead of $\int_I f\,dm$.
\remark If $f$ is continuous on $[a, b]$ then the Riemann and Lebesgue integrals agree.
\remark Most sets are {\em not} Borel sets.
\end{topic}

% 2.22
\begin{topic}{Sufficient Condition for Measure Zero}
\remark If $A \subseteq \mathbb{R}$ and every subset of $A$ is Lebesgue measurable then $m(A) = 0$.
\remark Every set of positive measure has unmeasurable subsets.
\end{topic}

% 2.23
\begin{topic}{Determinants}
\remark The $\Delta(T)$ in 2.20 is $|\det T\,|$.
\end{topic}


\material{Continuity Properties of Measurable Functions}

We assume in this section that $\mu$ is a measure on a locally compact Hausdorff space with the properties listed in 2.14 -- $\mu$ could be the Lebesgue measure on some $\mathbb{R}^k$.

% 2.24
\begin{topic}{Lusin's Theorem}
\remark $f$ complex measurable, $\mu(A) < \infty$, and $f = 0$ outside $A$. Then for $\epsilon > 0$ there exists $g \in C_c(X)$ with $$\mu(\{x : f(x) \neq g(x)\}) < \epsilon.$$ We may pick $g$ so that $$\sup_{x \in X} |g(x)| \leq \sup_{x \in X} |f(x)|.$$
\remark If $|f| \leq 1$ then there is a sequence $g_n \in C_c(X)$, $|g_n| \leq 1$, with $$f(x) = \lim_{n \to \infty} g_n(x)~~\text{a.e.}$$
\end{topic}

% 2.25
\begin{topic}{Vitali-Carath\'eodory Theorem}
\remark If $f \in L^1(\mu)$ is real valued and $\epsilon > 0$ then there exists $u$, upper semicontinuous and bounded from above, and $v$, lower semicontinuous and bounded from below, such that $u \leq f \leq v$ and $\int_X (v - u)\,d\mu < \epsilon$.
\end{topic}


\newpage
\section{$L^p$-Spaces}

\material{Convex Functions and Inequalities}

% 3.1
\begin{topic}{Definition}
\remark $\varphi$ is \term{convex} on $(a, b)$ if $x, y \in (a, b)$ and $\lambda \in [0, 1]$ imply $$\varphi((1 - \lambda)x + \lambda y) \leq (1 - \lambda)\varphi(x) + \lambda\varphi(y).$$ That is, the segment between $(x, \varphi(x))$ and $(y, \varphi(y))$ lies above the graph of $\varphi$.
\remark The above is equivalent to $a < s < t < u < b$ implying $$\dfrac{\varphi(t) - \varphi(s)}{t - s} \leq \dfrac{\varphi(u) - \varphi(t)}{u - t}.$$
\note The mean value theorem for differentiation with the above imply that $\varphi$, real differentiable, is convex in $(a, b)$ iff $a < s < t < b$ implies $\varphi^\prime(s) \leq \varphi^\prime(t)$.
\end{topic}

% 3.2
\begin{topic}{Convexity Implies Continuity}
\remark If $\varphi$ is convex on $(a, b)$ then $\varphi$ is continuous on $(a, b)$.
\note This relies on the fact that we are working on an {\em open} segment.
\end{topic}

% 3.3
\begin{topic}{Jensen's Inequality}
\remark $\mathfrak{M}$ a $\sigma$-algebra on $\Omega$, $\mu$ a positive measure on it such that $\mu(\Omega) = 1$. If $f \in L^1(\mu)$ is real with $f(\Omega) \subseteq (a, b)$ and $\varphi$ is convex on $(a, b)$ then $$\varphi\left(\int_\Omega f\,d\mu\right) \leq \int_\Omega (\varphi \circ f)\,d\mu.$$
\note $a = -\infty$ or $b = \infty$ are not excluded values.
\note If $\varphi \circ f \not\in L^1(\mu)$ then the integral has value $+\infty$ (see 1.31).
\example For $\varphi(x) = e^x$ we get $$\exp\left\{\int_\Omega f\,d\mu\right\} \leq \int_\Omega e^f\,d\mu.$$
\example If $\Omega = \{ p_1, \ldots, p_n \}$ and $\mu(\{p_i\}) = 1/n$, $f(p_i) = x_i$ then the example 1 becomes: $$\exp\left\{ \dfrac{1}{n}\sum_{i=1}^n x_i \right\} \leq \dfrac{1}{n}\sum_{i=1}^n e^{x_i}$$ for real $x_i$. Setting $y_i = e^{x_i}$ we can relate the arithmetic and geometric means of $n$ positive numbers: $$\left(\prod_{i=1}^n y_i\right)^{1/n} \leq \dfrac{1}{n}\sum_{i=1}^n y_i.$$ Given this, it is clear why $$\exp\left\{ \int_\Omega \log g\,d\mu \right\} \leq \int_\Omega g\,d\mu$$ are called the arithmetic and geometric means of the positive function $g$.
\example If $\mu(\{p_i\}) = \alpha_i > 0$ with $\sum \alpha_i = 1$ then we get a more general version of the above: $$\prod_{i=1}^n y_i^{\alpha_i} \leq \sum_{i=1}^n \alpha_i y_i.$$
\end{topic}

% 3.4
\begin{topic}{Definition}
\remark $p, q \in (1, \infty)$ are \term{conjugate exponents} if $p + q = pq$ (or, equivalently, $p^{-1} + q^{-1} = 1$).
\remark $p \to 1$ forces $q \to \infty$ and so $1$ and $\infty$ are regarded as conjugate exponents.
\remark Many denote $p$'s conjugate exponent by $p^\prime$.
\end{topic}

% 3.5
\begin{topic}{H\"older and Minkowski's Inequalities}
\setup $p$ and $q$ are conjugate exponents with $p \in (1, \infty)$ and $f$ and $g$ are measurable with range in $[0, \infty]$:
\remark H\"older's inequality: $$\int_X fg\,d\mu \leq \left\{\int_X f^p\,d\mu\right\}^{1/p} \left\{\int_X g^q\,d\mu\right\}^{1/q}.$$ If $p = q = 2$ then this is called Schwarz's inequality.
\remark Minkowski's inequality: $$\left\{\int_X (f+g)^p\,d\mu\right\}^{1/p} \leq \left\{\int_X f^p\,d\mu\right\}^{1/p} + \left\{\int_X g^p\,d\mu\right\}^{1/p}.$$
\note Assuming the right hand side of H\"older's inequality has only finite factors, equality holds if and only if there are constants $\alpha$ and $\beta$, not both zero, such that $\alpha f^p = \beta g^q$ a.e.
\end{topic}


\material{The $L^p$-spaces}
For this section, let $X$ be arbitrary and $\mu$ a positive measure.

% 3.6
\begin{topic}{Definition}
\remark If $0 \leq p \leq \infty$ and $f$ is a complex measurable function, then the \term{$L^p$-norm} of $f$ is $$\|f\|_p = \left\{\int_X |f|^p\,d\mu\right\}^{1/p}.$$ $L^p(\mu)$ is the collection of all $f$ for which $\|f\|_p < \infty$ and is called the \term{$L^p$-space} of $X$.
\remark If $u$ is the Lebesgue measure on $\mathbb{R}^k$ then we write $L^p(\mathbb{R}^k)$ instead of $L^p(\mu)$.
\remark If $\mu$ is the counting measure on a countable set $A$ we denote the $L^p$-space by $\ell^p(A)$ or just $\ell^p$. $x \in \ell^p$ is a sequence $x = \{\xi_n\}$ and $$\|x\|_p = \left\{\sum_{n=1}^\infty |\xi|^p\right\}^{1/p}.$$
\end{topic}

% 3.7
\begin{topic}{Definition}
\remark For $g : X \to [0, \infty]$ measurable, let $S$ be the set such that $\mu(g^{-1}((\alpha, \infty])) = 0$. If $S = \emptyset$ then set $\beta = \infty$, else $\beta = \inf S$. Since the countable union of sets of measure zero is a set of measure zero and $$g^{-1}((\beta, \infty]) = \bigcup_{n=1}^\infty g^{-1}\left(\left(\beta + \dfrac{1}{n}, \infty\right]\right),$$ $\beta \in S$. $\beta$ is the \term{essential supremum} of $g$.
\remark If $f$ is a complex measurable function then $\|f\|_\infty$ is the essential supremum of $|f|$. $L^\infty(\mu)$ is the set of all $f$ with $\|f\|_\infty < \infty$, it's members called the \term{essentially bounded} measurable functions on $X$.
\remark $L^\infty(\mathbb{R}^k)$ is the class of Lebesgue measure essentially bounded functions on $\mathbb{R}^k$.
\remark $\ell^\infty(A)$ is the class of bounded functions on $A$.
\note $|f(x)| \leq \lambda$ holds almost everywhere iff $\lambda \geq \|f\|_\infty$.
\end{topic}

% 3.8
\begin{topic}{H\"older's Inequality With $L^p$-norms}
\remark If $p$ and $q$ are conjugate exponents, $1 \leq p \leq \infty$, with $f \in L^p(\mu)$ and $g \in L^q(\mu)$ then $fg \in L^1(\mu)$ and $\|fg\|_1 \leq \|f\|_p \|g\|_q$.
\end{topic}

% 3.9
\begin{topic}{Minkowski's Inequality With $L-p$-norms}
\remark If $1 \leq p \leq \infty$ and $f, g \in L^p(\mu)$ then $f + g \in L^p(\mu)$ and $\|f + g\|_p \leq \|f\|_p + \|g\|_p$.
\end{topic}

% 3.10
\begin{topic}{Remarks}
\remark $L^p(\mu)$ is a complex vector space.
\remark Triangle inequality holds: $\|f - h\|_p \leq \|f - g\|_p + \|g - h\|_p$.
\note If $f \sim g$ (see 1.35) then $\|f - g\|_p = 0$.
\remark $L^p(\mu)$ is a complete metric space if we pass to equivalence classes under $\sim$.
\end{topic}

% 3.11
\begin{topic}{Completeness of $L^p(\mu)$}
\remark $L^p(\mu)$ is complete for every $1 \leq p \leq \infty$ and every positive measure $\mu$.
\end{topic}

% 3.12
\begin{topic}{Pointwise Convergence of Cauchy Subsequences}
\remark For $1 \leq p \leq \infty$ and $f_n \to f$ Cauchy in $L^p(\mu)$, $\{f_n\}$ has a subsequence converging pointwise to $f$ a.e.
\end{topic}

% 3.13
\begin{topic}{Density of (some) Simple Functions}
\remark For $1 \leq p < \infty$, the set of all complex, measurable, simple functions $s$ with $\mu(\{x : s(x) \neq 0\}) < \infty$ is dense in $L^p(\mu)$.
\end{topic}


\material{Approximation by Continuous Functions}
For this section $X$ is locally compact and Hausdorff, $\mu$ a measure on $\sigma$-algebra with the features in 2.14.

% 3.14
\begin{topic}{Density of $C_c(X)$}
\remark $C_c(X)$ is dense in $L^p(\mu)$ for $1 \leq p < \infty$.
\end{topic}

% 3.15
\begin{topic}{Remarks}
\remark $C_c(\mathbb{R}^k)$ has a metric that does not need to pass to equivalence classes.
\remark Likewise, the essential supremum there is the same as the supremum: $\|f\|_\infty = \sup_{x \in \mathbb{R}^k} |f(x)|$.
\remark If $1 \leq p < \infty$ then 3.14 gives $C_c(\mathbb{R}^k)$ is dense in $L^p(\mathbb{R}^k)$, which is complete by 3.11. $L_p(\mathbb{R}^k)$ is the completion of $C_c(\mathbb{R}^k)$ with respect to the $L^p(\mathbb{R}^k)$ metric.
\note Keep in mind that we are having {\em different} completions of the same set under different metrics.
\remark If the distance between $f, g \in C_c(\mathbb{R}^1)$ is given by $\int_{-\infty}^\infty |f(t) - g(t)|\,dt$ then the completion of the resulting metric space is the space of equivalence classes (under $\sim$) of Lebesgue integrable functions.
\note Important that the completion of functions on $\mathbb{R}^k$ are again functions on $\mathbb{R}^k$.
\remark The $L^\infty$-completion is $C_0(\mathbb{R}^k)$ of functions which vanish at infinity (see below).
\end{topic}

\noindent {\bf Reminder:} For 3.16 and 3.17, please remember in this section that $X$ is {\bf locally compact} and {\bf Hausdorff}.

% 3.16
\begin{topic}{Definition}
\remark The complex function $f$ \term{vanishes at infinity} if for $\epsilon > 0$ there is a $K$, compact, with $|f| < \epsilon$ on $K^c$.
\remark $C_0(X)$ is the class of all continuous functions $f$ on $X$ which vanish at infinity.
\remark $C_c(X) \subseteq C_0(X)$ with equality when $X$ is compact, in which case $C(X)$ is used for either.
\end{topic}

% 3.17
\begin{topic}{}
\remark $C_0(X)$ is the completion of $C_c(X)$ relative to the supremum norm metric: $\|f\| = \sup_{x \in X} |f(x)|$.
\end{topic}

\end{document}
